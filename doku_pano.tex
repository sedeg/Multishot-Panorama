\documentclass[liststotoc,bibtotoc,fontsize=14pt,]{scrreprt}
\usepackage[utf8]{inputenc} % Zeichenkodierung
\usepackage[ngerman]{babel} % neue deutsche Rechtschreibung
\usepackage{etoolbox}
\apptocmd{\thebibliography}{\raggedright}{}{}
\usepackage{graphicx}
\usepackage{url}
\usepackage[onehalfspacing]{setspace}
\usepackage{breakurl}
\usepackage{float}
\usepackage[table,xcdraw]{xcolor}
\usepackage{tabularx}
\usepackage[breaklinks]{hyperref}
\def\UrlBreaks{\do\/\do-}
\usepackage{tocloft}
\usepackage{chngcntr}
\usepackage{listings}
\usepackage{color}
\usepackage[parfill]{parskip}
\definecolor{lightgray}{rgb}{.9,.9,.9}
\definecolor{darkgray}{rgb}{.4,.4,.4}
\definecolor{purple}{rgb}{0.65, 0.12, 0.82}

\counterwithout{footnote}{chapter}

\deffootnote[2em]{2em}{2em}{%
	\makebox[2em][l]{\bfseries\thefootnotemark}}

\renewcommand{\cftchapdotsep}{\cftdotsep}
\renewcommand{\cftchapleader}{\cftdotfill{\cftchapdotsep}}
\usepackage{amsmath}
\usepackage[paper=a4paper,left=30mm,right=30mm,top=25mm,bottom=25mm]{geometry}
\usepackage[section]{placeins}
\usepackage[font=small,justification=justified]{caption}
\newcommand{\namesigdate}[3][Ort, Datum]{%
	\parbox{\textwidth}{
		\raggedleft #3 
		\vspace{2cm}
		
		\parbox{5cm}{
			\raggedright
			\rule{6cm}{1pt}\\
			#1 
		}
		\hfill
		\parbox{5cm}{
			\raggedright
			\rule{6cm}{1pt}\\
			#2
		}
	}
}


\newcommand*{\tabularwidth}{}
\newdimen\tabularwidth
\usepackage{minitoc}
\hypersetup{
	colorlinks,
	citecolor=black,
	filecolor=black,
	linkcolor=black,
	urlcolor=black
}


\title{Dokumentation Panoramafotografie}
\author{Sebastian Degner}

\begin{document}
	%\maketitle
	
	\begin{titlepage}
		\begin{center}
			\vspace{2cm}
			Dokumentation\\ \textbf{ Multishot-Technik in der digitalen Fotografie}\\ 
			\vspace{2,5cm}
			\includegraphics[width=5cm]{HTWK_Logo_RGB-transparent_250.png}\\
			
			\vspace{2,5cm}
			\huge \textbf{\textsf{Dokumentation Panoramafotografie}} \\
			\vspace{3cm}
			\fontsize{15}{18} \textbf{Hochschule für Technik, Wirtschaft und Kultur
				Leipzig\\ Fakultät Informatik, Mathematik und Naturwissenschaften\\   Masterstudiengang Medieninformatik}\\
			\vspace{3cm}
		\end{center}
		\normalsize{
			\begin{tabular}{ll}
				Eingereicht von: & {Sebastian Degner} \\
				 & {Sebastian Knabe} \\
				Studiengang: & 15 MIM\\
				Eingereicht am: & 09. Dezember 2016 \\
			\end{tabular}\\
		}
		
	\end{titlepage}
	
	
	
	
	
	\tableofcontents
	\clearpage
	\listoffigures
	\addcontentsline{toc}{chapter}{Abbildungsverzeichnis}

	\chapter{Einleitung}
	\label{ch:einleitung}
		
	\chapter{Vorbereitung}
	\label{ch:vorbereitung}
	
	\section{Verwendete Kameratechnik}
	\label{sec:technik}
	
	\section{Justage des NPP}
	\label{sec:npp}
	
	\chapter{Aufnahmen}
	\label{ch:aufnahmen}
	
	\section{180$^\circ$ Panorama -- Speck\grq s Hof}
	\label{sec:specks}

	\subsubsection{Aufnahmeort und -idee}
	Charakteristisch für die Leipziger Innenstadt, ist die Vielzahl an Passagen und Durchgangshöfen, welche das Stadtild prägen. Die älteste, erhaltene Ladenpassage ist Speck\grq s Hof, welche zwischen 1908 und 1929, durch den Bau des Messehauses entstand. Den Namen erhielt die Passage aufgrund des Kaufhofes des Freiherrn von Speck, welches zuvor an dieser Stelle stand. Sie befindet sich an der Kreuzung Reichstraße / Grimmaische Straße, in direkter Nähe zur Nikolaikirche und ist mit dem Hansa Haus verbunden. In den Jahren 1993 bis 1995 wurde die Passage aufwendig restauriert. Heute bietet Speck\grq s Hof eine architektonische und gestalterische Mischung aus Vergangenheit und Gegenwart und ist beispielsweise durch Malereien und Plastiken der Künstler Bruno Griesel, Johannes Grützke und Moritz Götze verziert.
	
	\bigskip
	
	
	\subsubsection{Kameraeinstellungen}
	
	
	\section{180$^\circ$ Panorama -- Auerbachskeller}
	\label{sec:auer}

	\section{180$^\circ$ Nachtpanorama -- Richard Wagner Platz}
	\label{sec:wagner}
	
	\section{360$^\circ$ Panorama -- Augustusplatz}
	\label{sec:augustus}
	
	\section{360$^\circ$ Nachtpanorama -- Marktplatz}
	\label{sec:markt}
	
	\section{Kugelpanorama --  Mädler-Passage}
	\label{sec:kugel}
	
	\chapter{Stitching}
	\label{ch:stitiching}
	Für ein Panorama aufgenommene Einzelbilder werden in der digitalen Fotografie mittels des sogenannten Stitchings zu einem Gesamtbild zusammengesetzt. Um diesen Vorgang weitestgehend zu automatisieren, gibt es verschiedene kostenlose oder auch kostenpflichtige Programme. In dieser Arbeit wird dabei näher auf Photoshop (\ref{sec:photoshop}) und PTGui Pro (\ref{sec:ptgui}) eingegangen, da diese für die Erstellung und Bearbeitung im Vordergrund stehen. Eine freie Softwarealternative für das Stitchen ist Hugin, welches in dieser Arbeit allerdings keine Verwendung findet, da die eben genannten Alternativen bereits alle benötigten Funktionen beherrschen.
	
	\section{PTGUI Pro}
	\label{sec:ptgui}
	
	Beim Start der Anwendung öffnet sich ein simpel gestaltetes Fenster, welches in Abb. \ref{img:ptgui_step_1} zu sehen ist. Die Software ist übersichtlich in Tabs aufgebaut, welche nach dem Laden der Fotos verschiedene Optionen bereitstellen. Mit einem Klick auf \grqq{}Advanced\grqq{} lassen sich diese noch erweitern.
	\begin{figure}[H]
		\includegraphics[width=\linewidth]{img/steps/PTGui_Step_1.PNG}
		\caption{PTGui Pro: Startfenster}
		\label{img:ptgui_step_1}
	\end{figure}
	\bigskip
	Über den \grqq{}Load images... -- Button\grqq{} lassen sich die Quellbilder zu einem neuen Projekt hinzufügen. Auch können an dieser Stelle Belichtungsreihen für die HDR--Entwicklung importiert werden. Anschließend werden die geladenen Bilder in dem bereits beschriebenen Startfenster angezeigt und die Einstellungsoptionen werden in Form von Tabs oberhalb angeordnet. Wie aus Abb. \ref{img:ptgui_step_2} weiter hervorgeht, erkennt PTGui automatisch die verwendete Brennweite und den Cropfaktor. Beide Angaben lassen sich zudem manuell einfügen. Unter dem Tab \grqq{}Source Images\grqq{} lässt sich ua. auch die Reihenfolge der Bilder festlegen.
	\begin{figure}[H]
		\includegraphics[width=\linewidth]{img/steps/PTGui_Step_2.PNG}
		\caption{PTGui Pro: Laden der Fotos}
		\label{img:ptgui_step_2}
	\end{figure}
	\bigskip
	Sind die erweiterten Einstellungen aktiviert, kann der Benutzer der Software zusätzliche manuelle, kameraspezifische Konfigurationen vornehmen s. Abb. \ref{img:ptgui_step_3}.
	\begin{figure}[H]
		\includegraphics[width=\linewidth]{img/steps/PTGui_Step_3.PNG}
		\caption{PTGui Pro: Objektiveinstellungen}
		\label{img:ptgui_step_3}
	\end{figure}
	\bigskip
	Mit einem Klick auf \grqq{}Align Images\grqq{} beginnt die Software mit dem Stitching der Einzelbilder. Anschließend öffnet sich ein weiteres Fenster, auf welchem das zusammengefügte Panorama betrachten lässt. Zudem werden den bereits bestehenden Tabs weitere angefügt s. Abb. \ref{img:ptgui_step_4}. 
	\begin{figure}[H]
		\includegraphics[width=\linewidth]{img/steps/PTGui_Step_4.PNG}
		\caption{PTGui Pro: Panoramaerstellung}
		\label{img:ptgui_step_4}
	\end{figure}
	\bigskip
	Wie in Abb. \ref{img:ptgui_step_4_1} zu sehen ist, kann man mit Hilfe der Maskierung in PTGui Pro, Bereiche kennzeichnen. Diese Markierungen geben an, ob diese Bereiche vom Foto im fertigen Panorama vorhanden (grün) oder nicht  vorhanden (rot) sein sollen. Die Software versucht diesem Wunsch nachzukommen und nimmt für rot markierte Bereiche überlappende Teile aus dem Nachbarbild und fügt diese stattdessen ein.
	\begin{figure}[H]
		\includegraphics[width=\linewidth]{img/steps/PTGui_Step_4_1.PNG}
		\caption{PTGui Pro: Maskierung}
		\label{img:ptgui_step_4_1}
	\end{figure}
	\bigskip
	Zudem bietet die Software die Möglichkeit für die jeweils zusammengefügten Fotos die sog. Kontrollpunkte zu überprüfen, verändern oder neue zu setzen. Mittels dieser bestimmt die Software, an welchen Stellen die beiden Bilder zusammengefügt werden. Dies ist in Abb. \ref{img:ptgui_step_5} zu sehen. Ein Kontrollpunkt auf dem linken Foto muss dabei einen auf dem rechten Bild entsprechen.
	\begin{figure}[H]
		\includegraphics[width=\linewidth]{img/steps/PTGui_Step_5.PNG}
		\caption{PTGui Pro: Kontrollpunkte}
		\label{img:ptgui_step_5}
	\end{figure}
	\bigskip
	Um die Änderung der Ausrichtung vom Bild zu übernehmen, muss der \grqq{}Optimizer\grqq{} im nächsten Tab ausgeführt werden s. Abb. \ref{img:ptgui_step_6}.
	\begin{figure}[H]
		\includegraphics[width=\linewidth]{img/steps/PTGui_Step_6.PNG}
		\caption{PTGui Pro: Optimierung}
		\label{img:ptgui_step_6}
	\end{figure}
	\bigskip
	Nach all diesen Schritten, kann das Panorama exportiert werden s. Abb. \ref{img:ptgui_step_7}. Für dieses Projekt wird das Panorama im TIFF-Format gespeichert, um es in der Nachbearbeitung möglichst verlustfrei korrigieren zu können. Durch einen Klick auf \grqq{}Create Panorama\grqq{} wird der Speichervorgang gestartet. Alternativ ist es auch möglich zusätzlich eine Projektdatei anzulegen, welche für eventuelle, nachträgliche Korrekturen geladen werden kann. 
	\begin{figure}[H]
		\includegraphics[width=\linewidth]{img/steps/PTGui_Step_7.PNG}
		\caption{PTGui Pro: Export}
		\label{img:ptgui_step_7}
	\end{figure}
	
	\section{Adobe Photoshop}
	\label{sec:photoshop}
	Adobe Photoshop beinhaltet das Tool \grqq{}Photomerge\grqq{}, mit dessen Hilfe Bilderreihen zu Panoramas zusammengefügt werden können. Dieses Kapitel befasst sich deswegen mit dem Stitching mittels Photoshop, wobei zugleich das Programm PTGui Pro (Kap. \ref{sec:ptgui}) als Vergleich herbeigezogen wird.
	\bigskip
	Das Tool Photomerge ist über das Menü unter Datei, Automatisieren zu erreichen. Nach dem Starten öffnet sich ein neues Fenster mit mehreren Einstellungsmöglichkeiten s. Abb. \ref{img:photoshop_step_2}.
	In dieser Ansicht kann auf der linken Seite der gewünschte Panoramatyp mittels einer Checkbox ausgewählt werden. In der Mitte befindet sich der Fotoimport. Es werden alle gängigen Bildformate unterstützt, auch RAW-Dateien. Zusätzliche Korrekturen, wie Vignettierungsentfernung, Verzerrungskorrektur und Auffüllung von Leerbereichen sind auswählbar. Im Gegensatz zu PTGui Pro sind keine weiteren Einstellungsmöglichkeiten vorhanden. Mit der Bestätigung auf \grqq{}OK\grqq{} erstellt Photomerge selbstständig und automatisiert ein Panorama.
	\begin{figure}[H]
		\includegraphics[width=\linewidth]{img/steps/Photoshop_Step_2.png}
		\caption{Photomerge: Startbildschirm}
		\label{img:photoshop_step_2}
	\end{figure}
	\bigskip

	Photomerge lädt die Fotos während des Stitchen als Ebenen in Photoshop und realisiert die Überblendungen der Einzelbilder mittels Ebenenmasken s. Abb. \ref{img:photoshop_step_3}. In diesem Beispiel handelt es sich um das 180 Grad Panorama vom Speck\grq s Hof (\ref{sec:specks}). Das Resultat ist einwandfrei und beinhaltet keinerlei Überblendungsfehler.
	\begin{figure}[H]
		\includegraphics[width=\linewidth]{img/steps/Photoshop_Step_3.png}
		\caption{Photomerge: Speck\grq s Hof}
		\label{img:photoshop_step_3}
	\end{figure}
	\bigskip
	Anders verhält es sich bei komplexeren Panoramen, wie z. B. bei dem Kugelpanorama der Mädler--Passage (s. Abb. \ref{img:photoshop_step_kugel}). Photomerge ist es nicht gelungen die Bilder in der richtigen Reihenfolge zusammenzusetzen. Da das Tool keine Möglichkeit einer Korrektur, Nachjustierung oder Neuanordnung bietet, kann das Panorama nicht damit umgesetzt werden.
	\begin{figure}[H]
		\includegraphics[width=\linewidth]{img/steps/Photoshop_Step_Kugel.png}
		\caption{Photomerge: Kugelpanorama}
		\label{img:photoshop_step_kugel}
	\end{figure}
	\bigskip
	Photoshop eignet sich daher für einfache Panoramen mit einer Ebene. Sobald die Anordnung der Fotos komplexer wird oder manuelle Korrekturen vorgenommen werden müssen, sollte auf eine Spezialsoftware zurückgegriffen werden.  
	
	\chapter{Nachbearbeitung}
	\label{ch:nach}
	Bei der Nachbearbeitung ist es wichtig darauf zu achten, möglichst verlustfreie Formate zu importieren. Auf diese Weise ist garantiert, dass die Bildqualität bei der Bearbeitung nicht so stark abnimmt, wie das bei den komprimierten Formaten, wie jpg der Fall ist. Bei diesem Projekt werden daher ausschließlich Panoramen im TIFF-Format bearbeitet und exportiert.
	
	\section{Adobe Photoshop}
	\label{sec:photo}
	Nach dem Export aus PTGui Pro, wird jedes Panorama mit Photoshop geöffnet. In dieser Software können störende Elemente, wie z. B. fehlende Pflastersteine oder Wolkenformationen, Blätter, Müll oder Personen entfernt werden. Dazu bietet Photoshop Tools, wie den Bereichsreparaturpinsel oder das Stempelwerkzeug. Diese sind in der Lage automatisiert inhaltsbasierte Korrekturen vorzunehmen. Auch wird bei diesem Vorgang bereits der Bildausschnitt gewählt und transparente Flächen abgeschnitten oder aufgefüllt.
	\bigskip
	Bei dem Kugelpanorama der Mädler--Passage (s. Abb. \ref{img:photoshop_step_kugel}) wird das Stativ durch das Überlagern eines stativfreien Fotos mittels Photoshop entfernt s. Abb. \ref{img:photoshop_bearbeitung}. Zudem wird die Spiegelung des Fotografen nachträglich entfernt.  
	
	\begin{figure}[H]
		\includegraphics[width=\linewidth]{img/steps/PS_Kugel_Edit.PNG}
		\caption{Photoshop: Nachbearbeitung}
		\label{img:photoshop_bearbeitung}
	\end{figure}
	
	
	\section{Adobe Lightroom}
	\label{sec:lightroom}
	Lightroom stammt auch aus dem Hause Adobe und dient hauptsächlich Verwaltung und Nachbearbeitung von Fotos. So erkennt die Software automatisch die meisten Kameramodelle und bietet entsprechende Entzerrungs- und Korrekturoptionen an. So ist beispielsweise möglich, Abberationen automatisiert zu entfernen. In diesem Projekt wird vorerst der Weißabgleich angepasst und anschließend die Tiefen und Höhen bearbeitet. Die Lichter und der Weißwert müssen so angeglichen werden, dass es möglichst keine ausgebrannten Stellen gibt. Zusätzlich erfolgt die Korrektur von Kontrast, Sättigung und Schärfe. in der Abbildung \ref{img:lightroom_bearbeitung} wird ein solcher Korrekturvorgang beispielhaft veranschaulicht. Zum Schluss erfolgt, falls nötig, nochmals die Auswahl des richtigen Bildausschnittes.
	\begin{figure}[H]
		\includegraphics[width=\linewidth]{img/steps/Lightroom_kugel.PNG}
		\caption{Lightroom: Nachbearbeitung}
		\label{img:lightroom_bearbeitung}
	\end{figure}

	
	\begin{thebibliography}{999}
	
		
	\end{thebibliography}
	
\end{document}