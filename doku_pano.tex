\documentclass[liststotoc,bibtotoc,fontsize=14pt,]{scrreprt}
\usepackage[utf8]{inputenc} % Zeichenkodierung
\usepackage[ngerman]{babel} % neue deutsche Rechtschreibung
\usepackage{etoolbox}
\apptocmd{\thebibliography}{\raggedright}{}{}
\usepackage{graphicx}
\usepackage{url}
\usepackage[onehalfspacing]{setspace}
\usepackage{breakurl}
\usepackage{float}
\usepackage[table,xcdraw]{xcolor}
\usepackage{tabularx}
\usepackage[breaklinks]{hyperref}
\def\UrlBreaks{\do\/\do-}
\usepackage{tocloft}
\usepackage{chngcntr}
\usepackage{listings}
\usepackage{color}
\usepackage[parfill]{parskip}
\definecolor{lightgray}{rgb}{.9,.9,.9}
\definecolor{darkgray}{rgb}{.4,.4,.4}
\definecolor{purple}{rgb}{0.65, 0.12, 0.82}

\counterwithout{footnote}{chapter}

\deffootnote[2em]{2em}{2em}{%
	\makebox[2em][l]{\bfseries\thefootnotemark}}

\renewcommand{\cftchapdotsep}{\cftdotsep}
\renewcommand{\cftchapleader}{\cftdotfill{\cftchapdotsep}}
\usepackage{amsmath}
\usepackage[paper=a4paper,left=30mm,right=30mm,top=25mm,bottom=25mm]{geometry}
\usepackage[section]{placeins}
\usepackage[font=small,justification=justified]{caption}
\newcommand{\namesigdate}[3][Ort, Datum]{%
	\parbox{\textwidth}{
		\raggedleft #3 
		\vspace{2cm}
		
		\parbox{5cm}{
			\raggedright
			\rule{6cm}{1pt}\\
			#1 
		}
		\hfill
		\parbox{5cm}{
			\raggedright
			\rule{6cm}{1pt}\\
			#2
		}
	}
}


\newcommand*{\tabularwidth}{}
\newdimen\tabularwidth
\usepackage{minitoc}
\hypersetup{
	colorlinks,
	citecolor=black,
	filecolor=black,
	linkcolor=black,
	urlcolor=black
}


\title{Dokumentation Panoramafotografie}
\author{Sebastian Degner}

\begin{document}
	%\maketitle
	
	\begin{titlepage}
		\begin{center}
			\vspace{2cm}
			Dokumentation\\ \textbf{ Multishot-Technik in der digitalen Fotografie}\\ 
			\vspace{2,5cm}
			\includegraphics[width=5cm]{HTWK_Logo_RGB-transparent_250.png}\\
			
			\vspace{2,5cm}
			\huge \textbf{\textsf{Dokumentation Panoramafotografie}} \\
			\vspace{3cm}
			\fontsize{15}{18} \textbf{Hochschule für Technik, Wirtschaft und Kultur
				Leipzig\\ Fakultät Informatik, Mathematik und Naturwissenschaften\\   Masterstudiengang Medieninformatik}\\
			\vspace{3cm}
		\end{center}
		\normalsize{
			\begin{tabular}{ll}
				Eingereicht von: & {Sebastian Degner} \\
				 & {Sebastian Knabe} \\
				Studiengang: & 15 MIM\\
				Eingereicht am: & 09. Dezember 2016 \\
			\end{tabular}\\
		}
		
	\end{titlepage}
	
	
	
	
	
	\tableofcontents
	\clearpage
	\listoffigures
	\addcontentsline{toc}{chapter}{Abbildungsverzeichnis}

	\chapter{Einleitung}
	\label{ch:einleitung}
		
	\chapter{Vorbereitung}
	\label{ch:vorbereitung}
	
	\section{Verwendete Kameratechnik}
	\label{sec:technik}
	
	\section{Justage des NPP}
	\label{sec:npp}
	
	\chapter{Aufnahmen}
	\label{ch:aufnahmen}
	
	\section{180$^\circ$ Panorama -- Speck\grq s Hof}
	\label{sec:specks}

	\subsubsection{Aufnahmeort und -idee}
	Charakteristisch für die Leipziger Innenstadt, ist die Vielzahl an Passagen und Durchgangshöfen, welche das Stadtild prägen. Die älteste, erhaltene Ladenpassage ist Speck\grq s Hof, welche zwischen 1908 und 1929, durch den Bau des Messehauses entstand. Den Namen erhielt die Passage aufgrund des Kaufhofes des Freiherrn von Speck, welches zuvor an dieser Stelle stand. Sie befindet sich an der Kreuzung Reichstraße / Grimmaische Straße, in direkter Nähe zur Nikolaikirche und ist mit dem Hansa Haus verbunden. In den Jahren 1993 bis 1995 wurde die Passage aufwendig restauriert. Heute bietet Speck\grq s Hof eine architektonische und gestalterische Mischung aus Vergangenheit und Gegenwart und ist beispielsweise durch Malereien und Plastiken der Künstler Bruno Griesel, Johannes Grützke und Moritz Götze verziert.
	
	\bigskip
	
	
	\subsubsection{Kameraeinstellungen}
	
	
	\section{180$^\circ$ Panorama -- Auerbachskeller}
	\label{sec:auer}

	\section{180$^\circ$ Nachtpanorama -- Richard Wagner Platz}
	\label{sec:wagner}
	
	\section{360$^\circ$ Panorama -- Augustusplatz}
	\label{sec:augustus}
	
	\section{360$^\circ$ Nachtpanorama -- Marktplatz}
	\label{sec:markt}
	
	\section{Kugelpanorama --  Mädler-Passage}
	\label{sec:kugel}
	
	\chapter{Stitching}
	\label{ch:stitiching}
	
	\section{Photoshop}
	\label{sec:photoshop}

	\section{PTGUI Pro}
	\label{sec:ptgui}
	
	\chapter{Nachbearbeitung}
	\label{ch:nach}
	
	\section{Lightroom}
	\label{sec:lightroom}
	
	\section{Photoshop}
	\label{sec:photo}
	

	
	\begin{thebibliography}{999}
	
		
	\end{thebibliography}
	
\end{document}