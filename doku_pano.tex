\documentclass[liststotoc,bibtotoc,fontsize=14pt,]{scrreprt}
\usepackage[utf8]{inputenc} % Zeichenkodierung
\usepackage[ngerman]{babel} % neue deutsche Rechtschreibung
\usepackage{etoolbox}
\apptocmd{\thebibliography}{\raggedright}{}{}
\usepackage{graphicx}
\usepackage{url}
\usepackage[onehalfspacing]{setspace}
\usepackage{breakurl}
\usepackage{float}
\usepackage[table,xcdraw]{xcolor}
\usepackage{tabularx}
\usepackage[breaklinks]{hyperref}
\def\UrlBreaks{\do\/\do-}
\usepackage{tocloft}
\usepackage{chngcntr}
\usepackage{listings}
\usepackage{color}
\definecolor{lightgray}{rgb}{.9,.9,.9}
\definecolor{darkgray}{rgb}{.4,.4,.4}
\definecolor{purple}{rgb}{0.65, 0.12, 0.82}

\counterwithout{footnote}{chapter}

\deffootnote[2em]{2em}{2em}{%
	\makebox[2em][l]{\bfseries\thefootnotemark}}

\renewcommand{\cftchapdotsep}{\cftdotsep}
\renewcommand{\cftchapleader}{\cftdotfill{\cftchapdotsep}}
\usepackage{amsmath}
\usepackage[paper=a4paper,left=30mm,right=30mm,top=25mm,bottom=25mm]{geometry}
\usepackage[section]{placeins}
\usepackage[font=small,justification=justified]{caption}
\newcommand{\namesigdate}[3][Ort, Datum]{%
	\parbox{\textwidth}{
		\raggedleft #3 
		\vspace{2cm}
		
		\parbox{5cm}{
			\raggedright
			\rule{6cm}{1pt}\\
			#1 
		}
		\hfill
		\parbox{5cm}{
			\raggedright
			\rule{6cm}{1pt}\\
			#2
		}
	}
}


\newcommand*{\tabularwidth}{}
\newdimen\tabularwidth
\usepackage{minitoc}
\hypersetup{
	colorlinks,
	citecolor=black,
	filecolor=black,
	linkcolor=black,
	urlcolor=black
}


\title{Dokumentation Panoramafotografie}
\author{Sebastian Degner}

\begin{document}
	%\maketitle
	
	\begin{titlepage}
		\begin{center}
			\vspace{2cm}
			Dokumentation\\ \textbf{ Multishot-Technik in der digitalen Fotografie}\\ 
			\vspace{2,5cm}
			\includegraphics[width=5cm]{HTWK_Logo_RGB-transparent_250.png}\\
			
			\vspace{2,5cm}
			\huge \textbf{\textsf{Dokumentation Panoramafotografie}} \\
			\vspace{3cm}
			\fontsize{15}{18} \textbf{Hochschule für Technik, Wirtschaft und Kultur
				Leipzig\\ Fakultät Informatik, Mathematik und Naturwissenschaften\\   Masterstudiengang Medieninformatik}\\
			\vspace{3cm}
		\end{center}
		\normalsize{
			\begin{tabular}{ll}
				Eingereicht von: & {Sebastian Degner} \\
				 & {Sebastian Knabe} \\
				Studiengang: & 15 MIM\\
				Eingereicht am: & 09. Dezember 2016 \\
			\end{tabular}\\
		}
		
	\end{titlepage}
	
	
	
	
	
	\tableofcontents
	\clearpage
	\listoffigures
	\addcontentsline{toc}{chapter}{Abbildungsverzeichnis}

	\chapter{Einleitung}
	\label{ch:einleitung}
		
	\chapter{Vorbereitung}
	\label{ch:vorbereitung}
	
	\section{Verwendete Kameratechnik}
	\label{sec:technik}
	
	\section{Justage des NPP}
	\label{sec:npp}
	
	\chapter{Aufnahmen}
	\label{ch:aufnahmen}
	
	\section{180$^\circ$ Panorama -- Speck\grq s Hof}
	\label{sec:specks}

	
	\section{180$^\circ$ Panorama -- Auerbachskeller}
	\label{sec:auer}

	\section{180$^\circ$ Nachtpanorama -- Richard Wagner Platz}
	\label{sec:wagner}
	
	\section{360$^\circ$ Panorama -- Augustusplatz}
	\label{sec:augustus}
	
	\section{360$^\circ$ Nachtpanorama -- Marktplatz}
	\label{sec:markt}
	
	\section{Kugelpanorama --  Mädler-Passage}
	\label{sec:kugel}
	
	\chapter{Stitching}
	\label{ch:stitiching}
	Für ein Panorama aufgenommene Einzelbilder werden in der digitalen Fotografie mittels des sogenannten Stitchings zu einem Gesamtbild zusammengesetzt. Um diesen Vorgang weitestgehend zu automatisieren, gibt es verschiedene kostenlose oder auch kostenpflichtige Programme. In dieser Arbeit wird dabei näher auf Photoshop (\ref{sec:photoshop}) und PTGui Pro (\ref{sec:ptgui}) eingegangen, da diese für die Erstellung und Bearbeitung im Vordergrund stehen. Eine freie Softwarealternative für das Stitchen ist Hugin, welches in dieser Arbeit allerdings keine Verwendung findet, da die eben genannten Alternativen bereits alle benötigten Funktionen beherrschen.
	
	\section{PTGUI Pro}
	\label{sec:ptgui}
	
	Beim Start der Anwendung öffnet sich ein simpel gestaltetes Fenster, welches in Abb. \ref{img:ptgui_step_1} zu sehen ist. Die Software ist übersichtlich in Tabs aufgebaut, welche nach dem Laden der Fotos verschiedene Optionen bereitstellen. Mit einem Klick auf \grqq{}Advanced\grqq{} lassen sich diese noch erweitern.
	\begin{figure}[H]
		\includegraphics[width=\linewidth]{img/steps/PTGui_Step_1.PNG}
		\caption{PTGui Pro: Startfenster}
		\label{img:ptgui_step_1}
	\end{figure}
	\bigskip
	Über den \grqq{}Load images... -- Button\grqq{} lassen sich die Quellbilder zu einem neuen Projekt hinzufügen. Auch können an dieser Stelle Belichtungsreihen für die HDR--Entwicklung importiert werden. Anschließend werden die geladenen Bilder in dem bereits beschriebenen Startfenster angezeigt und die Einstellungsoptionen werden in Form von Tabs oberhalb angeordnet. 
	\begin{figure}[H]
		\includegraphics[width=\linewidth]{img/steps/PTGui_Step_2.PNG}
		\caption{PTGui Pro: Laden der Fotos}
		\label{img:ptgui_step_2}
	\end{figure}
	\begin{figure}[H]
		\includegraphics[width=\linewidth]{img/steps/PTGui_Step_3.PNG}
		\caption{PTGui Pro: Objektiveinstellungen}
		\label{img:ptgui_step_3}
	\end{figure}
	\begin{figure}[H]
		\includegraphics[width=\linewidth]{img/steps/PTGui_Step_4.PNG}
		\caption{PTGui Pro: Maskierung}
		\label{img:ptgui_step_4}
	\end{figure}
	\begin{figure}[H]
		\includegraphics[width=\linewidth]{img/steps/PTGui_Step_5.PNG}
		\caption{PTGui Pro: Kontrollpunkte}
		\label{img:ptgui_step_5}
	\end{figure}
	\begin{figure}[H]
		\includegraphics[width=\linewidth]{img/steps/PTGui_Step_6.PNG}
		\caption{PTGui Pro: Optimierung}
		\label{img:ptgui_step_6}
	\end{figure}
	\begin{figure}[H]
		\includegraphics[width=\linewidth]{img/steps/PTGui_Step_7.PNG}
		\caption{PTGui Pro: Export}
		\label{img:ptgui_step_7}
	\end{figure}
	
	\section{Photoshop}
	\label{sec:photoshop}
	Stapelverarbeitung > Automatisch > fertig
	%TODO Bild
	
	\chapter{Nachbearbeitung}
	\label{ch:nach}
	
	\section{Lightroom}
	\label{sec:lightroom}
	
	\section{Photoshop}
	\label{sec:photo}
	

	
	\begin{thebibliography}{999}
	
		
	\end{thebibliography}
	
\end{document}